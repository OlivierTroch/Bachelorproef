%%=============================================================================
%% Voorwoord
%%=============================================================================

\chapter*{\IfLanguageName{dutch}{Woord vooraf}{Preface}}
\label{ch:voorwoord}

Deze bachelorproef ''Monitoring-tools voor Docker en Kubernetes'' werd geschreven met als doel het voltooien van mijn opleiding Toegepaste Informatica aan de Hogeschool Gent. Het onderwerp was voor mij ongezien en was daarom zeker een uitdaging, maar wetende dat Kubernetes, Docker en het monitoren ervan in deze tijden relevant zijn maakte mij alleen maar nieuwsgieriger. Dit onderwerp is tot stand gekomen door een docent in Hogeschool Gent om een onderdeel van het lessenpakket aan te vullen.

Graag zou ik ook even de tijd nemen om mensen te bedanken zonder wiens hulp deze bachelorproef niet tot stand zou gekomen zijn.

Eerst en vooral wil ik mijn promotor, die tegelijk mijn copromotor is, bedanken voor de nodige hulp. Door middel van goede en duidelijke feedback kon ik verbeteren waar nodig. Evenals de kans om dit onderwerp te mogen onderzoeken en documenteren.

Ook zou ik ook mijn ouders en medestudenten willen bedanken. In het bijzonder
Owen Van Damme, Emiel Van Belle en Rayen Nasra om mij the helpen met
morele steun en goede feedback doorheen deze periode.

Ten laatste zou ik ook Cedric Detemmerman willen bedanken voor een goede basis om op verder te bouwen bij het uitwerken van mijn onderwerp.

%% TODO:
%% Het voorwoord is het enige deel van de bachelorproef waar je vanuit je
%% eigen standpunt (``ik-vorm'') mag schrijven. Je kan hier bv. motiveren
%% waarom jij het onderwerp wil bespreken.
%% Vergeet ook niet te bedanken wie je geholpen/gesteund/... heeft

