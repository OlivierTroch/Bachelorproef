\chapter{Conclusie}
\label{ch:conclusie}

Op het begin van het onderzoek werden een aantal onderzoeksvragen opgesteld waaronder één hoofdonderzoeksvraag en meerdere deelvragen:

\begin{itemize}
    \item Welke monitoringtools zijn geschikt?
    \item Wat zijn de belangrijkste verschillen tussen de gekozen tools?
    \item Waar moet de gekozen tool inzicht op geven?
    \item Welk proof-of-concept is genoeg om de leerstof te verstaan?
\end{itemize}

De hoofdonderzoeksvraag was het onderzoeken welke tool potentieel geschikt was voor de studenten Toegepaste Informatica om aan te leren wat de mogelijkheden zijn bij het monitoren van een Docker/Kubernetes omgeving.  

Buiten de technische kant van het verhaal in een onderzoek naar tools was het ook belangrijke om te kijken naar de educatieve kant. Een volledig perfecte tool die moeilijk op te zetten is voor een student Toegepaste Informatica met een grote leercurve is minder geschikt dan een tool die het voor studenten haalbaarder maakt om te kunnen leren. 

Door de juiste analyse uit te voeren en de belangrijkste verschillen bloot te leggen, zijn voor het vinden van een gepaste monitoring tool, de volgende drie tools geschikt:

\begin{itemize}
    \item Sysdig
    \item Zabbix
    \item Prometheus
\end{itemize}

De drie tools waren praktisch hetzelfde waardoor de keuze werd gemaakt door de monitoringcommunity, waarbij de Prometheus tool met voorsprong uitblonk. Voor de Proof-of-Concept werd uiteindelijk voor Prometheus gekozen. 

De community van talloze experts die het opensource project ondersteunen zorgden ervoor dat de installatie van de tool vrij eenvoudig op te zetten was. Zonder deze tools was de moeilijkheidsgraad voor het opzetten van Prometheus een reden genoeg om een andere tool te kiezen. 

Prometheus is een befaamde tool die gebruikt wordt in vele bedrijven waardoor er vermoeden was dat deze tool zeker in de laatste vergelijking zou terecht komen. De tool heeft genoeg inzicht op alles wat nodig was om de onderzoeksvragen te beantwoorden.

Hopelijk wordt dit onderzoek gezien als een meerwaarde voor het lessenpakket bij de studenten Toegepaste Informatica zodat monitoring voor gecontaineriseerde omgevingen basiskennis wordt. Aangezien de populariteit van deze technologie enkel maar stijgt kan dit enkel maar een meerwaarde zijn, al is het enkel voor de lezer van dit onderzoek.



