\IfLanguageName{english}{%
\selectlanguage{dutch}
\chapter*{Samenvatting}

\selectlanguage{english}
}{}

\chapter*{\IfLanguageName{dutch}{Samenvatting}{Abstract}}

Toegepaste informatica in HoGent kon helaas niet beschikken over alle mogelijke technologieën die in de IT-wereld plaats vonden. Toch waren er bepaalde onderwerpen die de basiskennis van de studenten zeer goed konden versterken. Na het aanleren van Kubernetes en Docker was de monitoring van deze tools de logische volgende stap. Monitoring binnen de Kubernetes en Docker omgeving is om vele reden belangrijk en aan de hand van dit onderzoek werd onder andere een poging gedaan om dit te bewijzen.

De eerste stap in de bachelorproef was een situatie schetsen waarin de IT en HoGent zich toen bevonden. Het onderzoek focuste zich op het ontstaan van de virtualisatie en gecontaineriseerde omgeving en de bestaansreden van monitoring in zo een omgeving. 

Via de methodologie werd er onderzocht welke monitoringtools daadwerkelijk geschikt waren om te gebruiken binnen het lessenpakket. De MoSCoW-analyse, wat een requirements analyse was, was hiervoor de beste onderzoeksmethode. 

Na het vinden van een geschikte tool, werd hiervoor een gepaste Proof-of-Concept opgesteld. Een Proof-of-Concept die kon worden gerecreëerd door de studenten toepaste informatica, waarmee zij dan aan de slag kunnen om de wereld van monitoring te ontdekken. 

Uit het onderzoek voor het vinden van een geschikte monitoringtool waren drie tools overgebleven: Sysdig, Zabbix en Prometheus. Alle drie de tools voldeden op dezelfde manier aan de voorwaarden uit de requirements analyse. Uiteindelijk werd toch beslist om te kiezen voor Prometheus op vertrouwen van de community achter de monitoring tools. 