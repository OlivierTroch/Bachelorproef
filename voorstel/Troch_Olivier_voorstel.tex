%==============================================================================
% Sjabloon onderzoeksvoorstel bachelorproef
%==============================================================================
% Gebaseerd op LaTeX-sjabloon ‘Stylish Article’ (zie voorstel.cls)
% Auteur: Jens Buysse, Bert Van Vreckem
%
% Compileren in TeXstudio:
%
% - Zorg dat Biber de bibliografie compileert (en niet Biblatex)
%   Options > Configure > Build > Default Bibliography Tool: "txs:///biber"
% - F5 om te compileren en het resultaat te bekijken.
% - Als de bibliografie niet zichtbaar is, probeer dan F5 - F8 - F5
%   Met F8 compileer je de bibliografie apart.
%
% Als je JabRef gebruikt voor het bijhouden van de bibliografie, zorg dan
% dat je in ``biblatex''-modus opslaat: File > Switch to BibLaTeX mode.

\documentclass{voorstel}

\usepackage{lipsum}

%------------------------------------------------------------------------------
% Metadata over het voorstel
%------------------------------------------------------------------------------

%---------- Titel & auteur ----------------------------------------------------

% TODO: geef werktitel van je eigen voorstel op
\PaperTitle{Monitoring-tools voor Docker en Kubernetes }
\PaperType{Onderzoeksvoorstel Bachelorproef 2020-2021} % Type document

% TODO: vul je eigen naam in als auteur, geef ook je emailadres mee!
\Authors{Olivier Troch\textsuperscript{1}} % Authors
\CoPromotor{Bert Van Vreckem\textsuperscript{2} (HoGent)}
\affiliation{\textbf{Contact:}
    \textsuperscript{1} \href{mailto:olivier.troch.w2257@student.hogent.be}{olivier.troch.w2257@student.hogent.be};
    \textsuperscript{2} \href{mailto:bert.vanvreckem@hogent.be}{bert.vanvreckem@hogent.be};
}

%---------- Abstract ----------------------------------------------------------

\Abstract{
    Omdat binnen toegepaste informatica op HoGent geen installatie en gebruik is van monitoring tools, zullen aan de hand van deze bachelorproef verschillende monitoring tools binnen een omgeving getest worden om deze later te gebruiken voor studenten omtrent monitoring bij te leren. Door het willen invoegen van monitoring van containers (e.g. Docker) binnen deze opleiding moet op zoek gegaan worden naar een, bij voorkeur, open source monitoring tool. Om dit te onderzoeken zal er een reproduceerbaar 'proof-of-concept' worden opgesteld waarin meerdere, verschillende, monitoring tools zullen worden getest. In detail zal er in dit onderzoek voornamelijk gekeken worden naar het voortdurend meten van allerlei performantiemetrieken (CPU, I/O, enz.) en wat deze resultaten betekenen. Ook worden de verschillende framework-lagen vergeleken per monitoringtool. Er wordt verwacht om aan de hand van dit onderzoek een werkend en reproduceerbaar 'proof-of-concept' van geschikte open source monitoring-tools op te zetten, dat nadien kan worden gebruikt in het lessenpakket om studenten een basis te geven omtrent monitoring binnenin een docker omgeving. Aan de hand van dit werk zullen toekomstige studenten kunnen bijleren over monitoring.
}

%---------- Onderzoeksdomein en sleutelwoorden --------------------------------
% TODO: Sleutelwoorden:
%
% Het eerste sleutelwoord beschrijft het onderzoeksdomein. Je kan kiezen uit
% deze lijst:
%
% - Mobiele applicatieontwikkeling
% - Webapplicatieontwikkeling
% - Applicatieontwikkeling (andere)
% - Systeembeheer
% - Netwerkbeheer
% - Mainframe
% - E-business
% - Databanken en big data
% - Machineleertechnieken en kunstmatige intelligentie
% - Andere (specifieer)
%
% De andere sleutelwoorden zijn vrij te kiezen

\Keywords{Systeembeheer --- Docker --- Monitoring} % Keywords
\newcommand{\keywordname}{Sleutelwoorden} % Defines the keywords heading name

%---------- Titel, inhoud -----------------------------------------------------

\begin{document}
    
    \flushbottom % Makes all text pages the same height
    \maketitle % Print the title and abstract box
    \tableofcontents % Print the contents section
    \thispagestyle{empty} % Removes page numbering from the first page
    
    %------------------------------------------------------------------------------
    % Hoofdtekst
    %------------------------------------------------------------------------------
    
    % De hoofdtekst van het voorstel zit in een apart bestand, zodat het makkelijk
    % kan opgenomen worden in de bijlagen van de bachelorproef zelf.
    %---------- Inleiding ---------------------------------------------------------

\section{Introductie} % The \section*{} command stops section numbering
\label{sec:introductie}
Containertechnologie vindt in recordtempo zijn weg naar de data-omgeving van de ondernemingen.\\~\autocite{Cole2016}! Het gemak waarmee containerplatformen zoals Docker kunnen worden ingezet, suggereert dat ze de dominantste archictectuur zijn en zullen blijven voor deze en de volgende generatie services en microservices.
\\
De uitdaging om een goede monitoring van containers op te stellen is belangrijk. Zoals te verwachten is, zullen traditionele monitoringsplatforms die vooral gebasseerd zijn op virtualisatie, niet voldoende zijn. Containers zijn zeer vluchtig in onstaan en net zo snel in het verdwijnen. Die snelheid is vaak te danken aan een geautomatisseerd proces. Deze bevinden zich tussen host -en applicatielaag, wat het moeilijk maakt om in detail te zien hoe ze zich gedragen en of ze efficiënt gebruik maken van hulpbronnen (e.g. CPU, RAM, etc.) en goede systeem prestaties leveren. De valkuil voor vele bedrijven is dat deze denken dat omdat een container slechts een mini-host is, een eenvoudige hostmonitoring voldoende is. Dit idee wordt echter al snel uit de weg geruimd, omdat het aantal containers snel begint te vegroten wat een traditionele hostmonitoring niet kan bijhouden.
\\
\\
Binnen HoGent is er voorlopig geen opleidingsonderdeel omtrent monitoring, specifiek op een docker omgeving in een Kubernetes orkestratie. Daar dit een zeer interessant en relevant onderwerp is, is het een goede zaak om deze leerstof bij te leren aan de toekomstige studenten van HoGent. De onderzoeksvraag komt in dit geval uit van een docent tewerkgesteld in HoGent. Omdat er nog geen intern onderzoek heeft plaatsgevonden, is deze onderzoeksvraag dus onstaan. 
\\
De doelstelling van dit onderzoek is bepalen welke monitoringtool hiervoor geschikt is en welke haalbaar is voor de studenten. 
\\
Zo bekomen we volgende onderzoeksvragen:
\begin{itemize}
    \item Welke monitoringtools zijn geschikt?
    \item Wat zijn de belangrijkste verschillen tussen de gekozen tools?
    \item Waar moet de gekozen tool inzicht op geven?
    \item Welke alerts zijn belangrijk en welke niet?
    \item Welk proof-of-concept is genoeg om de leerstof te verstaan?
\end{itemize}

%---------- Stand van zaken ---------------------------------------------------

\section{State-of-the-art}
\label{sec:state-of-the-art}

Hoewel er vele onderzoeken zijn naar goede monitoringtools voor Docker, zijn het vooral onderzoeken waarbij ook de betalende tools vergeleken worden. \\~\autocite{Ribenzaft2020}! en ~\autocite{Cirelly2020}!\\ Deze onderzoeken specificeren zich ook niet op één specifieke 'proof-of-concept' maar leggen vooral de voor- en nadelen van elke tool uit.
Hoewel het boek van Alex Williams ~\autocite{2016}! ook niet echt een 'proof-of-concept' heeft, komt het onderzoek toch dichter in de buurt van wat het verwachte resultaat is van deze bachelorproef. Hierin wordt ook over een aantal andere onderwerpen gesproken omtrent monitoring die zeker interessant zijn. 
\\
\\
Daar een deel van dit boek ook een aantal tools beschrijft en vergelijkt is dit zeker relevant aan mijn onderzoeksvraag. Zoals het de bedoeling is van dit onderzoek om zowel zelf-gehoste open source-oplossingen als commerciële cloud-gebaseerde services te bekijken, kaart dit boek deze opties ook aan. Volgens het boek van Alex Williams~\autocite{2016}! is de conclusie dat de keuze sterk afhangt van de resultaten die je wenst te bereiken die bij uw werklast passen, eventueel met een combinatie van meerdere tools. Door opzoekingswerk en de verzameling van interne informatie om zo de vereisten te bepalen en te begrijpen, kan er gekozen worden voor de juiste monitoringtools. Tot dusver vind ik geen exacte kopie van mijn onderzoeksvraag, wat deze bachelorproef uniek maakt en dus interessanter.

% Voor literatuurverwijzingen zijn er twee belangrijke commando's:
% \autocite{KEY} => (Auteur, jaartal) Gebruik dit als de naam van de auteur
%   geen onderdeel is van de zin.
% \textcite{KEY} => Auteur (jaartal)  Gebruik dit als de auteursnaam wel een
%   functie heeft in de zin (bv. ``Uit onderzoek door Doll & Hill (1954) bleek

%---------- Methodologie ------------------------------------------------------
\section{Methodologie}
\label{sec:methodologie}

Aan de hand van een 'proof-of-concept' zal ik meerdere experimenten uitvoeren en documenteren. Ik zal uiteraard gebruik maken van Docker en Kubernetes. Aangezien ik er nog niet aan uit ben welke monitoringtools wel of niet de moeite zijn, want dit moet uitgewezen worden door mijn onderzoek, zal ik deze nog even achterwege laten. Maar om een goede conclusie te maken zal er zeker met genoeg tools geëxperimenteerd worden. Eventueel kan er een vragenlijst opgesteld worden om studenten te bevragen wat voor hun belangrijk is bij het gebruiken van een tool.

%---------- Verwachte resultaten ----------------------------------------------
\section{Verwachte resultaten}
\label{sec:verwachte_resultaten}

Uit deze studie naar een gepaste monitoringtool wordt verwacht dat aan de hand van mijn experimenten, het duidelijk wordt welke tool een basis kan zijn voor een deel van het opleidingsonderdeel waarop beslist wordt wat toekomstige studenten kunnen gebruiken. Het opstellen van een goede proof-of-concept is het belangrijkste voor het bereiken van goede resultaten.

%---------- Verwachte conclusies ----------------------------------------------
\section{Verwachte conclusies}
\label{sec:verwachte_conclusies}

Het experiment zou een goede basis moeten hebben om zo de essentiële onderdelen van monitoring aan te leren aan studenten toegepaste informatica. Met een basis wordt vooral bedoeld dat de 'proof-of-concept' makkelijk reproduceerbaar is en dat de 'beste' monitoringtool en zijn functionaliteiten allemaal uit te voeren zijn. Bovendien is het de bedoeling dat de 'proof-of-concept' alle deelaspecten van monitoring bevat met inzicht op, in geval van dit onderzoek, zicht op het voortdurend meten van allerlei performantiemetrieken. Indien het experiment aan deze eisen voldoet, mogen we concluderen dat dit een geslaagd onderzoek is.


    
    %------------------------------------------------------------------------------
    % Referentielijst
    %------------------------------------------------------------------------------
    % TODO: de gerefereerde werken moeten in BibTeX-bestand ``voorstel.bib''
    % voorkomen. Gebruik JabRef om je bibliografie bij te houden en vergeet niet
    % om compatibiliteit met Biber/BibLaTeX aan te zetten (File > Switch to
    % BibLaTeX mode)
    
    \phantomsection
    \printbibliography[heading=bibintoc]
    
\end{document}